% =============================================================
% Define some variables
% =============================================================
% Report
\newcommand{\reportTitle}{Contrôle robuste d'un micro-capteur inertiel}
\newcommand{\reportSubject}{Contrôle robuste d'un micro-capteur inertiel}
\newcommand{\reportType}{Rapport Final - Stage de fin d'études}
\newcommand{\reportShortName}{TFE 2017}
\newcommand{\reportDates}{01-04-2017 - 29-09-2017}
\newcommand{\reportAuthor}{Dehaeze Thomas}

% Author
\newcommand{\authorFirstName}{Thomas}
\newcommand{\authorLastName}{Dehaeze}
\newcommand{\authorEmail}{dehaeze.thomas@gmail.com}

% School
\newcommand{\schoolName}{\'Ecole Centrale de Lyon}
\newcommand{\schoolPlace}{\'Ecully}
\newcommand{\schoolLogo}{logo-ecl.pdf}

% School Tutor
\newcommand{\schoolTutorFirstName}{Anton}
\newcommand{\schoolTutorLastName}{Korniienko}
\newcommand{\schoolTutorEmail}{anton.korniienko@ec-lyon.fr}

% Company
\newcommand{\companyName}{Asygn}
\newcommand{\companyPlace}{Montbonnot-Saint-Martin}
\newcommand{\companyLogo}{logo-asygn.pdf}

% Company Tutors
\newcommand{\companyTutorFirstName}{Christophe}
\newcommand{\companyTutorLastName}{Le Blanc}
\newcommand{\companyTutorEmail}{christophe.leblanc@asygn.com}
% =============================================================


% =============================================================
% Graphic Path
% =============================================================
\graphicspath{%
    {../ressources/}%
    {../ressources/pdf/}%
    {../ressources/images/}%
    {../ressources/logos/}%
    {../ressources/tikz/}%
}
% =============================================================


% =============================================================
% Colors
% =============================================================
\usepackage{xcolor}% Color extension

\definecolor{colorblack}{rgb}{0, 0, 0}
\definecolor{colorblue}{rgb}{0, 0.4470, 0.7410}
\definecolor{colorred}{rgb}{0.8500, 0.3250, 0.0980}
\definecolor{coloryellow}{rgb}{0.9290, 0.6940, 0.1250}
\definecolor{colorpurple}{rgb}{0.4940, 0.1840, 0.5560}
\definecolor{colorgreen}{rgb}{0.4660, 0.6740, 0.1880}
\definecolor{colorcyan}{rgb}{0.3010, 0.7450, 0.9330}
\definecolor{colorbordeau}{rgb}{0.6350, 0.0780, 0.1840}

% Main color
\definecolor{maincolor}{RGB}{89, 9, 38}
\definecolor{secondcolor}{RGB}{20, 9, 89}
% =============================================================


% =============================================================
% Hyperref and PDF Options
% =============================================================
\usepackage[                    % setup the hyperref-package options
    pdftitle={\reportTitle},    % title (PDF meta)
    pdfauthor={\reportAuthor},  % author (PDF meta)
    pdfsubject={\reportSubject},% subject (PDF meta)
    pdfcreator={\reportAuthor}, % creator (PDF meta)
    plainpages=false,           %
    pdfborder={0 0 0},          %
    breaklinks=true,            % allow line break inside links
    bookmarksnumbered=true,     %
    bookmarksopen=true,         %
    colorlinks=true,            % colorize links?
    allcolors=colorblack,%
]{hyperref}

\providecommand\phantomsection{} % TODO: explain
% =============================================================


% =============================================================
% Some variable to customize theme
% =============================================================
% toogletrue to have a "fancy chapter" tooglefalse to don't
\newtoggle{fancychapter}
\toggletrue{fancychapter}

\newlength\titleindent{}
\setlength\titleindent{4em}

\newtoggle{sectionmargin}
\toggletrue{sectionmargin}
% =============================================================

